% !TeX spellcheck = sk_SK-Slovak
\documentclass[a4paper]{article}
\usepackage[slovak]{babel}
\usepackage[utf8]{inputenc}
\usepackage[T1]{fontenc}
\usepackage{a4wide}
\usepackage{amsmath}
\usepackage{amsfonts}
\usepackage{amsthm,amssymb}
\usepackage{mathrsfs}
\usepackage[small,bf]{caption}
\usepackage{subcaption}
\usepackage{xcolor}
\usepackage{graphicx}
\usepackage{enumerate}
\usepackage{hyperref}
\usepackage[a4paper, total={7in, 10.2in}]{geometry}



\pagestyle{empty}
\setlength{\parindent}{0pt}

\newenvironment{modenumerate}
{\enumerate\setupmodenumerate}
{\endenumerate}

%\renewcommand{\thesubsection}{\thesection.\alph{subsection}}
\renewcommand{\thesubsection}{\alph{subsection})}


\begin{document} 
	
	\pagenumbering{arabic}
	\pagestyle{plain}
	
	\begin{center}
		\sc\large
		PHYSICAL BASED ANIMATIONS AND MATHEMATICAL MODELING HW 2 
		\\
		SAT 
	\end{center}
	
	Autor: Marián Kravec
	\\
	\\
	Na vstupe máme dvojicu objektov daných geometriou a našou úlohou je tieto objekty obaliť tak aby sa s nimi lepšie pracovalo a následne zistiť, či existuje deliaca os, čiže os na ktorej obrazy našich objektov budú mať prázdny prienik. Ak takáto os existuje, môžeme tvrdiť, že objekty nie sú v kolízii, ak takúto os nenájdeme, znamená to, že obálky našich objektov sú v kolízii a na určenie kolízie samotných objektov potrebujeme použiť inú metódy (a iné, presnejšie obálky).
	\\
	\\
	Náš prvý objekt je objekt $A$ daný takouto geometriou: $\{(0;0;10), (0;5;0), (5;0;0), (-5;-5;0)\}$. Pre tento objekt chceme  vypočítať obálku AABB. To spravíme v časti c).
	\\
	\\
	Náš druhý objekt, čiže objekt $B$ je daný geometriou: ${(50;50;50), (30+x;30+y;25), (25;30;30), (35;30;30)}$. Ako prvý krok si musíme dopočítať koeficienty $x$ a $y$, keďže tieto koeficienty sa počítajú z dátumu narodenia (konkrétne dňa) v mojom prípade $18.9 \implies x=1, y=8$, takže výsledná geometria objektu $B$ je ${(50;50;50), (31;38;25), (25;30;30), (35;30;30)}$. Tento objekt chceme obaliť do OBB obálky. Preto prvý krok výpočtu obálky je výpočet kovariančnej matice tohto objektu.
	
	\subsection{}
	Keďže sú naše objekty trojrozmerné naša kovariančná matica bude rozmerov $3x3$ pričom bude vyzerať následovne:
	\begin{align*}
		K = \begin{pmatrix}
			cov(x,x) & cov(y,x) & cov(z,x) \\
			cov(x,y) & cov(y,y) & cov(z,y) \\
			cov(x,z) & cov(y,z) & cov(z,z)
		\end{pmatrix}
		\\
		\text{(avšak vieme, že platí $cov(x,y)=cov(y,x)$ tak všetky}
		\\
		\text{kovariancie zoradíme v lexikografickom poradí)}
		\\
		K = \begin{pmatrix}
			cov(x,x) & cov(x,y) & cov(x,z) \\
			cov(x,y) & cov(y,y) & cov(y,z) \\
			cov(x,z) & cov(y,z) & cov(z,z)
		\end{pmatrix}
	\end{align*}

	Vidíme, že potrebujeme vypočítať 6 rôznych kovariancii. Vieme, že vzorec pre výpočet odhadu kovariancie je:
	\\ 
	$cov(x,y)=\frac{1}{n}\sum_{i=1}^{n}(x_i-\bar{x})(y_i-\bar{y})$
	\\
	($\bar{x}$ označuje aritmetický priemer všetkých vstupných $x$ čiže $\bar{x}=\frac{1}{n}\sum_{i=1}^{n}x_i$) 
	(v tomto prípade ide o takzvanú Population Covariance, v štatistike sa často používa aj Sample Covariance ktorá má vzorec $cov(x,y)=\frac{1}{n-1}\sum_{i=1}^{n}(x_i-\bar{x})(y_i-\bar{y})$, čiže sa líši iba tým, že v menovateli namiesto $n$ je $n-1$ vďaka takejto úprave by mala byť Sample Covariance nevychýleným odhadom skutočnej kovariancie, avšak na cvičeniach sme využívali vzorec pre Population Covariance preto som ho použil v mojom riešení)
	\\
	Z tohto vzorca nám vyplýva, že ako prvé potrebujeme vypočítať aritmetický priemer pre $x, y, z$:
	
	\begin{align*}
		\bar{x}=\frac{1}{n}\sum_{i=1}^{n}x_i=\frac{1}{4}(50+31+25+35)=\frac{1}{4}141=35.25
		\\
		\bar{y}=\frac{1}{n}\sum_{i=1}^{n}y_i=\frac{1}{4}(50+38+30+30)=\frac{1}{4}148=37
		\\
		\bar{z}=\frac{1}{n}\sum_{i=1}^{n}z_i=\frac{1}{4}(50+25+30+30)=\frac{1}{4}135=33.75
	\end{align*}
	\newpage
	Teraz môžeme vypočítať jednotlivé kovariancie:
	
	\begin{align*}
		&cov(x,x)=\frac{1}{n}\sum_{i=1}^{n}(x_i-\bar{x})(x_i-\bar{x})=
		\\
		&=\frac{1}{4}((50-35.25)(50-35.25)+(31-35.25)(31-35.25)+
		\\
		&+(25-35.25)(25-35.25)+(35-35.25)(35-35.25))=
		\\
		&=\frac{1}{4}((14.75)(14.75)+(-4.25)(-4.25)+
		(-10.25)(-10.25)+(-0.25)(-0.25))=
		\\
		&=\frac{1}{4}(217.5625+18.0625+105.0625+0.0625)
		\\
		&=\frac{1}{4}340.75=85.1875
		\\
		\\
		&cov(x,y)=\frac{1}{n}\sum_{i=1}^{n}(x_i-\bar{x})(y_i-\bar{y})=
		\\
		&=\frac{1}{4}((50-35.25)(50-37)+(31-35.25)(38-37)+
		\\
		&+(25-35.25)(30-37)+(35-35.25)(30-37))=
		\\
		&=\frac{1}{4}((14.75)(13)+(-4.25)(1)+
		(-10.25)(-7)+(-0.25)(-7))=
		\\
		&=\frac{1}{4}(191.75-4.25+71.75+1.75)
		\\
		&=\frac{1}{4}261=65.25
		\\
		\\
		&cov(x,z)=\frac{1}{n}\sum_{i=1}^{n}(x_i-\bar{x})(z_i-\bar{z})=
		\\
		&=\frac{1}{4}((50-35.25)(50-33.75)+(31-35.25)(25-33.75)+
		\\
		&+(25-35.25)(30-33.75)+(35-35.25)(30-33.75))=
		\\
		&=\frac{1}{4}((14.75)(16.25)+(-4.25)(-8.75)+
		(-10.25)(-3.75)+(-0.25)(-3.75))=
		\\
		&=\frac{1}{4}(239.6875+37.1875+38.4375+0.9375)
		\\
		&=\frac{1}{4}316.25=79.0625
	\end{align*}
	\begin{align*}
		&cov(y,y)=\frac{1}{n}\sum_{i=1}^{n}(y_i-\bar{y})(y_i-\bar{y})=
		\\
		&=\frac{1}{4}((50-37)(50-37)+(38-37)(38-37)+
		\\
		&+(30-37)(30-37)+(30-37)(30-37))=
		\\
		&=\frac{1}{4}((13)(13)+(1)(1)+
		(-7)(-7)+(-7)(-7))=
		\\
		&=\frac{1}{4}(169+1+49+49)
		\\
		&=\frac{1}{4}268=67
		\\
		\\
		&cov(y,z)=\frac{1}{n}\sum_{i=1}^{n}(y_i-\bar{y})(z_i-\bar{z})=
		\\
		&=\frac{1}{4}((50-37)(50-33.75)+(38-37)(25-33.75)+
		\\
		&+(30-37)(30-33.75)+(30-37)(30-33.75))=
		\\
		&=\frac{1}{4}((13)(16.25)+(1)(-8.75)+
		(-7)(-3.75)+(-7)(-3.75))=
		\\
		&=\frac{1}{4}(211.25-8.75+26.25+26.25)
		\\
		&=\frac{1}{4}255=63.75
		\\
		\\
		&cov(z,z)=\frac{1}{n}\sum_{i=1}^{n}(z_i-\bar{z})(z_i-\bar{z})=
		\\
		&=\frac{1}{4}((50-33.75)(50-33.75)+(25-33.75)(25-33.75)+
		\\
		&+(30-33.75)(30-33.75)+(30-33.75)(30-33.75))=
		\\
		&=\frac{1}{4}((16.25)(16.25)+(-8.75)(-8.75)+
		(-3.75)(-3.75)+(-3.75)(-3.75))=
		\\
		&=\frac{1}{4}(264.0625+76.5625+14.0625+14.0625)
		\\
		&=\frac{1}{4}368.75=92.1875
	\end{align*}
	Teraz môžeme vypočítané hodnoty vložiť do kovariančnej matice:
	\begin{align*}
		K_B = \begin{pmatrix}
			cov(x,x) & cov(x,y) & cov(x,z) \\
			cov(x,y) & cov(y,y) & cov(y,z) \\
			cov(x,z) & cov(y,z) & cov(z,z)
		\end{pmatrix} =
		\begin{pmatrix}
			85.1875 & 65.25 & 79.0625 \\
			65.25 & 67 & 63.75 \\
			79.0625 & 63.75 & 92.1875
		\end{pmatrix}
	\end{align*}
	\newpage
	\subsection{}
	Teraz potrebujeme vypočítať vlastné hodnoty a vlastné vektory tejto matice. Čiže hľadáme hodnoty $\lambda$ a vektory $\vec{d}$ pre ktoré platí takýto vzťah:
	$K \vec{d}=\lambda \vec{d}$. Tento vzťah prepísať ako $(K-\lambda I) \vec{d} = \vec{0}$ pričom vieme, že táto rovnica má netriviálne riešenie ($\vec{d} \neq \vec{0}$) práve vtedy keď $(K-\lambda I)$ je singulárna matica, čiže vtedy keď $det(K-\lambda I)=0$.
	\\
	Takže prvým krokom je nájsť hodnoty $\lambda$ pre ktoré je daná matica singulárna.
	\begin{align*}
		&det(K_B-\lambda^B I)=0
		\\
		&det(
		\begin{pmatrix}
			85.1875 & 65.25 & 79.0625 \\
			65.25 & 67 & 63.75 \\
			79.0625 & 63.75 & 92.1875
		\end{pmatrix}
		-	
		\begin{pmatrix}
			\lambda & 0 & 0 \\
			0 & \lambda & 0 \\
			0 & 0 & \lambda
		\end{pmatrix})=0
		\\		
		&det(
		\begin{pmatrix}
			85.1875-\lambda & 65.25 & 79.0625 \\
			65.25 & 67-\lambda & 63.75 \\
			79.0625 & 63.75 & 92.1875-\lambda
		\end{pmatrix})=0
		\\
		&\text{Za normálnych okolností by sme na výpočet}
		\\
		&\text{tohto determinantu použili všeobecný vzorec:}
		\\
		&det(A)=\sum_{i=1}^{n}a_{ij}(-1)^{i+j}A_{ij}
		\\
		&\text{Ale keďže naša matica je $3x3$ a počítame}
		\\
		&\text{to ručne použijeme Sarrusovo pravidlo:}
		\\
		&det(A)=a_{11}a_{22}a_{33}+a_{21}a_{32}a_{13}+a_{31}a_{12}a_{23}-
		\\
		&-a_{13}a_{22}a_{31}-a_{23}a_{23}a_{11}-a_{33}a_{12}a_{21}
		\\
		&\text{Po aplikácii vzorca na našu maticu dostaneme:}
		\\
		&(85.1875-\lambda)(67-\lambda)(92.1875-\lambda)+(65.25)(63.75)(79.0625)+
		\\
		&+(79.0625)(65.25)(63.75)-(79.0625)(67-\lambda)(79.0625)-
		\\
		&-(63.75)(63.75)(85.1875-\lambda)-(92.1875-\lambda)(65.25)(65.25)=0
		\\
		\\
		&(5707.5625-85.1875\lambda-67\lambda+\lambda^2)(92.1875-\lambda)+328875.29296875+
		\\
		&+328875.29296875-(418808.88671875-6250.87890625\lambda)-
		\\
		&-(346207.32421875-4064.0625\lambda)-(392494.04296875-4257.5625\lambda)=0
		\\
		\\
		&526165.91796875-5707.5625\lambda-14029.78515625\lambda+152.1875\lambda^2+92.1875\lambda^2-\lambda^3+
		\\
		&+328875.29296875+328875.29296875-418808.88671875+6250.87890625\lambda-
		\\
		&-346207.32421875+4064.0625\lambda-392494.04296875+4257.5625\lambda=0
		\\
		\\
		&26406.25-5164.84375\lambda+244.375\lambda^2-\lambda^3=0
		\\
		&(-1)\lambda^3+244.375\lambda^2+(-5164.84375)\lambda+26406.25=0
		\\
	\end{align*}
	\begin{align*}
		&\text{Dostali sme rovnicu tretieho stupňa ktorej korene sú naše hľadané vlastné hodnoty.}
		\\
		&\text{Na výpočet koreňov použijeme vzorce pre kubickú rovnicu}
		\\
		&\text{Najskôr si vypočítame medzihodnoty aby finálny výpočet bol čo najjednoduchší}
		\\
		&\text{Ako prvú vypočítame hodnotu $q$:}
		\\
		&q=\frac{bc}{6a^2}-\frac{d}{2a}-\frac{b^3}{27a^3}
		\\
		&q=\frac{(244.375)(-5164.84375)}{(6)(-1)^2}-\frac{26406.25}{(2)(-1)}-\frac{(244.375)^3}{(27)(-1)^3}
		\\
		&q=\frac{-1262158.69140625}{6}-\frac{26406.25}{-2}-\frac{14593864.990234375}{-27}
		\\
		&q=-210359.78190104166+13203.125+540513.5181568287
		\\
		&q=343356.8612557871
		\\
		&\text{Ďalej vypočítame hodnoty $p_1$ a $p_2$:}
		\\
		&p_1=\sqrt[3]{q+\sqrt{q^2+\left(\frac{c}{3a}-\frac{b^2}{9a^2}\right)^3}}
		\\
		&p_2=\sqrt[3]{q-\sqrt{q^2+\left(\frac{c}{3a}-\frac{b^2}{9a^2}\right)^3}}
		\\
		&p_1=\sqrt[3]{343356.8612557871+\sqrt{(343356.8612557871)^2+\left(\frac{-5164.84375}{(3)(-1)}-\frac{(244.375)^2}{(9)(-1)^2}\right)^3}}
		\\
		&p_1=\sqrt[3]{343356.8612557871+\sqrt{117893934171.42581+\left(\frac{-5164.84375}{-3}-\frac{59719.140625}{9}\right)^3}}
		\\
		&p_1=\sqrt[3]{343356.8612557871+\sqrt{117893934171.42581+\left(1721.6145833333333-6635.460069444444\right)^3}}
		\\
		&p_1=\sqrt[3]{343356.8612557871+\sqrt{117893934171.42581+\left(-4913.845486111111\right)^3}}
		\\
		&p_1=\sqrt[3]{343356.8612557871+\sqrt{117893934171.42581-118649110971.76732}}
		\\
		&p_1=\sqrt[3]{343356.8612557871+\sqrt{-755176800.341507}}
		\\
		&p_1=\sqrt[3]{343356.8612557871+27480.480351360435i}
		\\
		&p_1=70.07398835693077 +1.865915824425889i
		\\
		&\text{$p_2$ je odlišné od $p_1$ iba v jednom znamienku, inak je výpočet totožný preto rovnaké kroky preskočíme.}
		\\
		&p_2=\sqrt[3]{343356.8612557871-27480.480351360435i}
		\\
		&p_2=70.07398835693077 -1.865915824425889 i
		\\
		&\text{Zároveň vieme, že hodnoty $\omega$ a $\omega^2$ (jednotkové korene) sú:}
		\\
		&\omega=\frac{-1+i\sqrt{3}}{2}
		\\
		&\omega^3=\frac{-1-i\sqrt{3}}{2}
		\\
	\end{align*}
	\begin{align*}
		&\text{Samotné vzorce pre korene sú následovné:}
		\\
		&\lambda_1=p_1+p_2-\frac{b}{3a}
		\\
		&\lambda_2=\omega p_1 +\omega^2 p_2 -\frac{b}{3a}
		\\
		&\lambda_3=\omega^2 p_1 +\omega p_2-\frac{b}{3a}
		\\
		\\		
		&\text{Tieto vzorce teraz aplikujeme na náš prípad:}
		\\
		&\lambda_1^B=70.07398835693077 +1.865915824425889 i+70.07398835693077 -1.865915824425889 i-
		\frac{244.375}{(3)(-1)}
		\\
		&\lambda_1^B=140.1479767138614-
		\frac{244.375}{-3}
		\\
		&\lambda_1^B=140.1479767138614+81.45833333333333
		\\
		&\lambda_1^B=221.6063100471947
		\\
		&\lambda_2^B=(70.07398835693077 +1.865915824425889 i)\left(\frac{-1+i\sqrt{3}}{2}\right)+
		\\
		&+(70.07398835693077 -1.865915824425889 i)\left(\frac{-1-i\sqrt{3}}{2}\right)-
		\frac{244.375}{(3)(-1)}
		\\
		&\lambda_2^B=\left(\frac{-70.07398835693077+121.37170812319404 i -1.865915824425889 i -3.2318610105524086}{2}\right)+
		\\
		&+\left(\frac{-70.07398835693077-121.37170812319404 i +1.865915824425889 i -3.2318610105524086}{2}\right)-
		\frac{244.375}{(3)(-1)}
		\\
		&\lambda_2^B=(-36.65292468374159+59.75289614938408i)+
		\\
		&+(-36.65292468374159-59.75289614938408i)-
		\frac{244.375}{(3)(-1)}
		\\
		&\lambda_2^B=-73.30584936748318-
		\frac{244.375}{-3}
		\\
		&\lambda_2^B=-73.30584936748318+81.45833333333333
		\\
		&\lambda_2^B=8.152483965850152
		\\
		&\lambda_3^B=(70.07398835693077 +1.865915824425889 i)\left(\frac{-1-i\sqrt{3}}{2}\right)+
		\\
		&+(70.07398835693077 -1.865915824425889 i)\left(\frac{-1+i\sqrt{3}}{2}\right)-
		\frac{244.375}{(3)(-1)}
		\\
		&\lambda_3^B=\left(\frac{-70.07398835693077-121.37170812319404 i -1.865915824425889 i +3.2318610105524086}{2}\right)+
		\\
		&+\left(\frac{-70.07398835693077+121.37170812319404 i +1.865915824425889 i +3.2318610105524086}{2}\right)-
		\frac{244.375}{(3)(-1)}
		\\
		&\lambda_3^B=(-33.421063673189189-61.61881197380996i)+
		\\
		&+(-33.42106367318918+61.61881197380996i)-
		\frac{244.375}{(3)(-1)}
		\\
		&\lambda_3^B=-66.84212734637836-
		\frac{244.375}{-3}
		\\
		&\lambda_3^B=-66.84212734637836+81.45833333333333
		\\
		&\lambda_3^B=14.616205986954967
	\end{align*}
	Takže naše výsledné vlastné čísla sú (po zaokrúhlení na 3 desatinné miesta, pre ďalšie výpočty budú použité v čo najpresnejšom tvare):
	$\lambda_1^B=221.606, \lambda_2^B=8.152, \lambda_3^B=14.616$
	\\
	\newpage
	Teraz potrebujeme k týmto hodnotám dopočítať k ním prislúchajúce vlastné vektory. Ako sme definovali skôr, vieme, že platí následujúci vzťah: $(K-\lambda I) \vec{d} = \vec{0}$. Takže na to aby sme našli vlastný vektor potrebujeme nájsť riešenie tohto systému rovníc, preto si ho prepíšeme do tvaru v ktorom sa s ním bude lepšie pracovať.
	\begin{align*}
		&\text{Klasický zápis sústavy rovníc:}
		\\
		&\begin{pmatrix}
			85.1875-\lambda & 65.25 & 79.0625 \\
			65.25 & 67-\lambda & 63.75 \\
			79.0625 & 63.75 & 92.1875-\lambda
		\end{pmatrix}
		\begin{pmatrix}
			d_1 \\
			d_2\\
			d_3
		\end{pmatrix}=0
		\\
		&\text{Namiesto neho použijeme na zápis rozšírenú maticu:}
		\\
		&\left( \mkern1mu \begin{array}{@{}ccc|c@{}}
			85.1875-\lambda & 65.25 & 79.0625 & 0\\
			65.25 & 67-\lambda & 63.75 & 0\\
			79.0625 & 63.75 & 92.1875-\lambda & 0
		\end{array} \mkern1mu \right)
		\\
		&\text{Keďže vieme, že matica $(K-\lambda I)$ je singulárna preto vieme, že tento náš systém }
		\\
		&\text{bude mať nekonečne veľa riešení pričom všetky riešenia sa dajú zapísať ako $\alpha \vec{d}$ kde$\alpha \in \mathbb{R}$.}
		\\
		&\text{Preto môžeme uvažovať, že $d_1$ bude vždy 1 a odpočítať koeficient pri $d_1$ od oboch strán rovnice}
		\\
		&\text{a riešiť sústavu troch rovníc o dvoch neznámych pričom vieme, že systém je lineárne závislí. }
		\\
		&\text{Výsledná matica vyzerá následovne:}
		\\
		&\left( \mkern1mu \begin{array}{@{}cc|c@{}}
			65.25 & 79.0625 & -85.1875+\lambda \\
			67-\lambda & 63.75 & -65.25  \\
			63.75 & 92.1875-\lambda & -79.0625 
		\end{array} \mkern1mu \right)
		\\
		&\text{Na výpočet koreňa použije Gaussovu eliminačnú metódu.}
		\\
		&\text{Začnime s výpočtom vektora $d_1^B$ prislúchajúceho vlastnej hodnote $\lambda_1^B$}
		\\
		&\left( \mkern1mu \begin{array}{@{}cc|c@{}}
			65.25 & 79.0625 & -85.1875+\lambda_1^B \\
			67-\lambda_1^B & 63.75 & -65.25  \\
			63.75 & 92.1875-\lambda_1^B & -79.0625 
		\end{array} \mkern1mu \right)
		\\
		&\sim\left( \mkern1mu
		\begin{array}{@{}cc|c@{}}
			65.25 & 79.0625 & -85.1875+221.6063100471947 \\
			67-221.6063100471947 & 63.75 & -65.25\\
			63.75 & 92.1875-221.6063100471947 & -79.0625 
		\end{array} \mkern1mu \right)\sim
		\\
		&\sim\left( \mkern1mu
		\begin{array}{@{}cc|c@{}}
			65.25 & 79.0625 & 136.4188100471947\\
			-154.6063100471947 & 63.75 & -65.25 \\
			63.75 & -129.4188100471947 & -79.0625 
		\end{array} \mkern1mu \right)\sim
		\\
		&\sim\left( \mkern1mu
		\begin{array}{@{}cc|c@{}}
			1 & 1.2116858237547892 & 2.090709732524057\\
			-154.6063100471947 & 63.75 & -65.25 \\
			63.75 & -129.4188100471947 & -79.0625 
		\end{array} \mkern1mu \right)\sim
		\\
		&\sim\left( \mkern1mu
		\begin{array}{@{}cc|c@{}}
			1 & 1.2116858237547892 & 2.090709732524057\\
			0 & 251.08427414722345 & 257.9869171253019\\
			0 & -206.6637813115625 & -212.34524544840863
		\end{array} \mkern1mu \right)\sim
		\\
		&\sim\left( \mkern1mu
		\begin{array}{@{}cc|c@{}}
			1 & 1.2116858237547892 & 2.090709732524057\\
			0 & 1 & 1.027491339318332\\
			0 & 1 & 1.027491339318334
		\end{array} \mkern1mu \right)\sim
		\\
		&\sim\left( \mkern1mu
		\begin{array}{@{}cc|c@{}}
			1 & 0 & 0.845713042641212\\
			0 & 1 & 1.027491339318332\\
			0 & 1 & 1.027491339318334
		\end{array} \mkern1mu \right)
		\\
		&\tilde{d}_1^B=(1, 0.845713042641212, 1.027491339318332)
		\\
		&\text{Ako posledný krok vlastný vektor normalizujeme (vynásobíme tak aby mal dĺžku 1)}
		\\
		&d_1^B=\frac{\tilde{d}_1^B}{|\tilde{d}_1^B|}=
		\frac{(1, 0.845713042641212, 1.027491339318332)}{\sqrt{1^2 + 0.845713042641212^2 + 1.027491339318332^2}}=
		\\
		&=\frac{(1, 0.845713042641212, 1.027491339318332)}{\sqrt{1 + 0.7152305504934565 + 1.0557384523741797}}
		=\frac{(1, 0.845713042641212, 1.027491339318332)}{\sqrt{2.7709690028676364}}=
		\\
		&=\frac{(1, 0.845713042641212, 1.027491339318332)}{1.664622780952981}
		=(0.6007367022981083, 0.5080508643267811, 0.6172517588219614)
	\end{align*}
	\begin{align*}
		&\left( \mkern1mu \begin{array}{@{}cc|c@{}}
			65.25 & 79.0625 & -85.1875+\lambda_2^B \\
			67-\lambda_2^B & 63.75 & -65.25  \\
			63.75 & 92.1875-\lambda_2^B & -79.0625 
		\end{array} \mkern1mu \right)
		\\
		&\sim\left( \mkern1mu
		\begin{array}{@{}cc|c@{}}
			65.25 & 79.0625 & -85.1875+8.152483965850152 \\
			67-8.152483965850152 & 63.75 & -65.25\\
			63.75 & 92.1875-8.152483965850152 & -79.0625 
		\end{array} \mkern1mu \right)\sim
		\\
		&\sim\left( \mkern1mu
		\begin{array}{@{}cc|c@{}}
			65.25 & 79.0625 & -77.03501603414985\\
			58.84751603414985 & 63.75 & -65.25 \\
			63.75 & 84.03501603414985 & -79.0625 
		\end{array} \mkern1mu \right)\sim
		\\
		&\sim\left( \mkern1mu
		\begin{array}{@{}cc|c@{}}
			1 & 1.2116858237547892 & -1.1806132725540206\\
			58.84751603414985 & 63.75 & -65.25 \\
			63.75 & 84.03501603414985 & -79.0625 
		\end{array} \mkern1mu \right)\sim
		\\
		&\sim\left( \mkern1mu
		\begin{array}{@{}cc|c@{}}
			1 & 1.2116858237547892 & -1.1806132725540206\\
			0 & -7.554700941762022 & 4.226158486752851\\
			0 & 6.7900447697820425 & -3.798403874681185
		\end{array} \mkern1mu \right)\sim
		\\
		&\sim\left( \mkern1mu
		\begin{array}{@{}cc|c@{}}
			1 & 1.2116858237547892 & -1.1806132725540206\\
			0 & 1 & -0.5594077805768394\\
			0 & 1 & -0.5594077805768447
		\end{array} \mkern1mu \right)\sim
		\\
		&\sim\left( \mkern1mu
		\begin{array}{@{}cc|c@{}}
			1 & 0 & -0.5027867951309346\\
			0 & 1 & -0.5594077805768394\\
			0 & 1 & -0.5594077805768447
		\end{array} \mkern1mu \right)
		\\
		&\tilde{d}_2^B=(1, -0.5027867951309346, -0.5594077805768394)
		\\
		&d_2^B=\frac{\tilde{d}_2^B}{|\tilde{d}_2^B|}=
		\frac{(1, -0.5027867951309346, -0.5594077805768394)}{\sqrt{1^2 + (-0.5027867951309346)^2 + (-0.5594077805768394)^2}}=
		\\
		&=\frac{(1, -0.5027867951309346, -0.5594077805768394)}{\sqrt{1 + 0.2527945613580364 + 0.31293706496990525}}
		=\frac{(1, -0.5027867951309346, -0.5594077805768394)}{\sqrt{1.5657316263279417}}=
		\\
		&=\frac{(1, -0.5027867951309346, -0.5594077805768394)}{1.2512919828433098}
		=(0.7991739847383189, -0.40181412653859777, -0.4470641450972119)
	\end{align*}
	\begin{align*}
		&\left( \mkern1mu \begin{array}{@{}cc|c@{}}
			65.25 & 79.0625 & -85.1875+\lambda_3^B \\
			67-\lambda_3^B & 63.75 & -65.25  \\
			63.75 & 92.1875-\lambda_3^B & -79.0625 
		\end{array} \mkern1mu \right)
		\\
		&\sim\left( \mkern1mu
		\begin{array}{@{}cc|c@{}}
			65.25 & 79.0625 & -85.1875+14.616205986954967 \\
			67-14.616205986954967 & 63.75 & -65.25\\
			63.75 & 92.1875-14.616205986954967 & -79.0625 
		\end{array} \mkern1mu \right)\sim
		\\
		&\sim\left( \mkern1mu
		\begin{array}{@{}cc|c@{}}
			65.25 & 79.0625 & -70.57129401304503\\
			52.38379401304503 & 63.75 & -65.25 \\
			63.75 & 77.57129401304503 & -79.0625 
		\end{array} \mkern1mu \right)\sim
		\\
		&\sim\left( \mkern1mu
		\begin{array}{@{}cc|c@{}}
			1 & 1.2116858237547892 & -1.0815523986673568\\
			52.383794013045035 & 63.75 & -65.25 \\
			63.75 & 77.57129401304503 & -79.0625 
		\end{array} \mkern1mu \right)\sim
		\\
		&\sim\left( \mkern1mu
		\begin{array}{@{}cc|c@{}}
			1 & 1.2116858237547892 & -1.0815523986673568\\
			0 & 0.27729939990233277 & -8.594181933894426\\
			0 & 0.326322748677228 & -10.113534584956014
		\end{array} \mkern1mu \right)\sim
		\\
		&\sim\left( \mkern1mu
		\begin{array}{@{}cc|c@{}}
			1 & 1.2116858237547892 & -1.0815523986673568\\
			0 & 1 & -30.992428894261476\\
			0 & 1 & -30.992428894252487
		\end{array} \mkern1mu \right)\sim
		\\
		&\sim\left( \mkern1mu
		\begin{array}{@{}cc|c@{}}
			1 & 0 & 36.47153433623759\\
			0 & 1 & -30.992428894261476\\
			0 & 1 & -30.992428894252487
		\end{array} \mkern1mu \right)
		\\
		&\tilde{d}_3^B=(1, 36.47153433623759, -30.992428894261476)
		\\
		&d_3^B=\frac{\tilde{d}_3^B}{|\tilde{d}_3^B|}=
		\frac{(1, 36.47153433623759, -30.992428894261476)}{\sqrt{1^2 + 36.47153433623759^2 + (-30.992428894261476)^2}}=
		\\
		&=\frac{(1, 36.47153433623759, -30.992428894261476)}{\sqrt{1 + 1330.1728168393574 + 960.5306487658536}}
		=\frac{(1, 36.47153433623759, -30.992428894261476)}{\sqrt{2291.703465605211}}=
		\\
		&=\frac{(1, 36.47153433623759, -30.992428894261476)}{47.87173973865177}
		=(0.02088915099930236, 0.7618593879259077, -0.6474055270073694)
	\end{align*}
	Čiže naše vlastné vektory sú (po zaokrúhlení na 3 desatinné miesta):$d_1^B=(0.601, 0.508, 0.617)$, 
	\\$d_2^B=(0.799, -0.402, -0.447)$ a $d_3^B=(0.021, 0.762, -0.647)$.
	\newpage
	Predtým ako sa pustíme do počítanie samotných obálok skontrolujeme, že či nami získané hodnoty a vektory sú skutočne vlastnými hodnotami a vlastnými vektormi našej kovariačnej matice.
	Budeme testovať či platí $K d=\lambda d$ (strany rovnice si budeme označovať ako $LS$ a $PS$ a budeme skúmať či $LS=PS$).
	\begin{align*}
		&\text{Začnime s $d_1^B$ a $\lambda_1^B$:}
		\\
		&LS=K_B d_1^B= \begin{pmatrix}
			85.1875 & 65.25 & 79.0625 \\
			65.25 & 67 & 63.75 \\
			79.0625 & 63.75 & 92.1875
		\end{pmatrix}
		\begin{pmatrix}
			0.6007367022981083 \\ 
			0.5080508643267811 \\ 
			0.6172517588219614
		\end{pmatrix}=
		\\
		&=\begin{pmatrix}
			(85.1875)(0.6007367022981083) + (65.25)(0.5080508643267811) + (79.0625)(0.6172517588219614) \\
			(65.25)(0.6007367022981083) + (67)(0.5080508643267811) + (63.75)(0.6172517588219614) \\
			(79.0625)(0.6007367022981083) + (63.75)(0.5080508643267811) + (92.1875)(0.6172517588219614)
		\end{pmatrix}=
		\\
		&=\begin{pmatrix}
			51.1752578270201 + 33.15031889732247 + 48.80146718186132 \\
			39.19806982495157 + 34.03940790989434 + 39.34979962490004 \\
			47.495745525444185 + 32.3882426008323 + 56.902896516399565
		\end{pmatrix}=
		\begin{pmatrix}
			133.12704390620388 \\
			112.58727735974594 \\
			136.78688464267606
		\end{pmatrix}
		\\
		&PS=\lambda_1^B d_1^B= 221.6063100471947 \begin{pmatrix}
			0.6007367022981083 \\ 
			0.5080508643267811 \\ 
			0.6172517588219614
		\end{pmatrix} = 
		\begin{pmatrix}
			133.12704390620388 \\
			112.58727735974591 \\
			136.78688464267583
		\end{pmatrix}
		\\
		&LS \approxeq PS \text{  }\qedsymbol
		\\
		&\text{Pre $d_1^B$ a $\lambda_1^B$ skúška správnosti potvrdila, že ide o vlastné číslo a vlastný vektor} 
		\\
		&\text{(malé rozdiely nanajvýš okolo $10^{-12}$ sú spôsobené nevyhnutným zaokrúhľovaním)}
		\\
		&\text{Teraz zopakujeme skúšku správnosti ešte pre dvojicu $d_2^B$ a $\lambda_2^B$ a dvojicu $d_3^B$ a $\lambda_3^B$} 
		\\
		&LS=K_B d_2^B= \begin{pmatrix}
			85.1875 & 65.25 & 79.0625 \\
			65.25 & 67 & 63.75 \\
			79.0625 & 63.75 & 92.1875
		\end{pmatrix}
		\begin{pmatrix}
			0.7991739847383189  \\
			-0.40181412653859777 \\
			-0.4470641450972119
		\end{pmatrix}=
		\\
		&=\begin{pmatrix}
			(85.1875)(0.7991739847383189) + (65.25)(-0.40181412653859777) + (79.0625)(-0.4470641450972119) \\
			(65.25)(0.7991739847383189) + (67)(-0.40181412653859777) + (63.75)(-0.4470641450972119) \\
			(79.0625)(0.7991739847383189) + (63.75)(-0.40181412653859777) + (92.1875)(-0.4470641450972119)
		\end{pmatrix}=
		\\
		&=\begin{pmatrix}
			68.07963382489554 -26.218371756643503 -35.34600897174831 \\
			52.14610250417531 -26.92154647808605 -28.500339249947256 \\
			63.184693168373336 -25.61565056683561 -41.21372587614922
		\end{pmatrix}=
		\begin{pmatrix}
			6.515253096503722 \\
			-3.275783223857996 \\
			-3.644683274611488
		\end{pmatrix}
		\\
		&PS=\lambda_2^B d_2^B= 8.152483965850152 \begin{pmatrix}
			0.7991739847383189  \\
			-0.40181412653859777 \\
			-0.4470641450972119
		\end{pmatrix} = 
		\begin{pmatrix}
			6.515253096503719 \\
			-3.2757832238580025 \\
			-3.6446832746115256
		\end{pmatrix}
		\\
		&LS \approxeq PS \text{  }\qedsymbol
		\\
		&LS=K_B d_3^B= \begin{pmatrix}
			85.1875 & 65.25 & 79.0625 \\
			65.25 & 67 & 63.75 \\
			79.0625 & 63.75 & 92.1875
		\end{pmatrix}
		\begin{pmatrix}
			0.02088915099930236  \\
			0.7618593879259077  \\
			-0.6474055270073694
		\end{pmatrix}=
		\\
		&=\begin{pmatrix}
			(85.1875)(0.02088915099930236) + (65.25)(0.7618593879259077) + (79.0625)(-0.6474055270073694) \\
			(65.25)(0.02088915099930236) + (67)(0.7618593879259077) + (63.75)(-0.6474055270073694) \\
			(79.0625)(0.02088915099930236) + (63.75)(0.7618593879259077) + (92.1875)(-0.6474055270073694)
		\end{pmatrix}=
		\\
		&=\begin{pmatrix}
			1.7794945507530697 + 49.71132506216548 -51.185499479020145 \\
			1.3630171027044788 + 51.04457899103581 -41.2721023467198 \\
			1.6515485008823427 + 48.568535980276614 -59.68269702099187
		\end{pmatrix}=
		\begin{pmatrix}
			0.3053201338984053 \\
			11.135493747020497 \\
			-9.46261253983291
		\end{pmatrix}
		\\
		&PS=\lambda_3^B d_3^B= 14.616205986954967 \begin{pmatrix}
			0.02088915099930236  \\
			0.7618593879259077  \\
			-0.6474055270073694
		\end{pmatrix} = 
		\begin{pmatrix}
			0.30532013389840945 \\
			11.135493747020499 \\
			-9.462612539832849
		\end{pmatrix}
		\\
		&LS \approxeq PS \text{  }\qedsymbol
		\\
	\end{align*}
	Všetky skúšky správnosti potvrdili, že naše hodnoty a vektory sú naozaj vlastnými vektormi a vlastnými hodnotami našej kovariančnej matice.
	\newpage
	\subsection{}
	Teraz  keď už máme vlastné hodnoty a vektory pre objekt B môžeme začať s výpočtami obálok našich objektov.
	
	Ako prvé začnime s objektom A. Pre tento objekt chceme obálku AABB, o tejto obálke vieme, že jej vlastné vektory (respektíve vektory ktorými je daná) sú štandardné jednotkové vektory čiže: $d_1^A = e_1 = (1, 0, 0)$, $d_2^A = e_2 = (0, 1, 0)$, $d_3^A = e_3 = (0, 0, 1)$. 
	Ta takže ako ďalšie potrebujeme vypočítať rozmery našej obálky a jej ťažisko (stred), na všetky tieto hodnoty potrebujeme zistiť minimálne a maximálne hodnoty zobrazení bodov z našej geometrie na naše vlastné vektory:
	\begin{align*}
		&MIN_1^A = min\{d_1^A A_1, d_1^A A_2, d_1^A A_3, d_1^A A_4\} = min\{(1, 0, 0) (0, 0, 10), (1, 0, 0) (0, 5, 0), (1, 0, 0) (5, 0, 0), (1, 0, 0) (-5, -5, 0)\}=
		\\
		&= min\{((1)(0) + (0)(0) + (0)(10)), ((1)(0) + (0)(5) + (0)(0)), ((1)(5) + (0)(0) + (0)(0)), ((1)(-5) + (0)(-5) + (0)(0))\}=
		\\
		&=min\{(0+0+0), (0+0+0), (5+0+0), (-5+0+0)\}
		=min\{0, 0, 5, -5\} = -5
		\\
		&MAX_1^A = max\{d_1^A A_1, d_1^A A_2, d_1^A A_3, d_1^A A_4\} = max\{(1, 0, 0) (0, 0, 10), (1, 0, 0) (0, 5, 0), (1, 0, 0) (5, 0, 0), (1, 0, 0) (-5, -5, 0)\}=
		\\
		&= max\{((1)(0) + (0)(0) + (0)(10)), ((1)(0) + (0)(5) + (0)(0)), ((1)(5) + (0)(0) + (0)(0)), ((1)(-5) + (0)(-5) + (0)(0))\}=
		\\
		&=max\{(0+0+0), (0+0+0), (5+0+0), (-5+0+0)\}
		=max\{0, 0, 5, -5\} = 5
		\\
		&MIN_2^A = min\{d_2^A A_1, d_2^A A_2, d_2^A A_3, d_2^A A_4\} = min\{(0, 1, 0) (0, 0, 10), (0, 1, 0) (0, 5, 0), (0, 1, 0) (5, 0, 0), (0, 1, 0) (-5, -5, 0)\}=
		\\
		&= min\{((0)(0) + (1)(0) + (0)(10)), ((0)(0) + (1)(5) + (0)(0)), ((0)(5) + (1)(0) + (0)(0)), ((0)(-5) + (1)(-5) + (0)(0))\}=
		\\
		&=min\{(0+0+0), (0+5+0), (0+0+0), (0-5+0)\}
		=min\{0, 5, 0, -5\} = -5
		\\
		&MAX_2^A = max\{d_2^A A_1, d_2^A A_2, d_2^A A_3, d_2^A A_4\} = max\{(0, 1, 0) (0, 0, 10), (0, 1, 0) (0, 5, 0), (0, 1, 0) (5, 0, 0), (0, 1, 0) (-5, -5, 0)\}=
		\\
		&= max\{((0)(0) + (1)(0) + (0)(10)), ((0)(0) + (1)(5) + (0)(0)), ((0)(5) + (1)(0) + (0)(0)), ((0)(-5) + (1)(-5) + (0)(0))\}=
		\\
		&=min\{(0+0+0), (0+5+0), (0+0+0), (0-5+0)\}
		=min\{0, 5, 0, -5\} = -5
		\\
		&MIN_3^A = min\{d_3^A A_1, d_3^A A_2, d_3^A A_3, d_3^A A_4\} = min\{(0, 0, 1) (0, 0, 10), (0, 0, 1) (0, 5, 0), (0, 0, 1) (5, 0, 0), (0, 0, 1) (-5, -5, 0)\}=
		\\
		&= min\{((0)(0) + (0)(0) + (1)(10)), ((0)(0) + (0)(5) + (1)(0)), ((0)(5) + (0)(0) + (1)(0)), ((0)(-5) + (0)(-5) + (1)(0))\}=
		\\
		&=min\{(0+0+10), (0+0+0), (0+0+0), (0+0+0)\}
		=min\{10, 0, 0, 0\} = 0
		\\
		&MAX_3^A = max\{d_3^A A_1, d_3^A A_2, d_3^A A_3, d_3^A A_4\} = max\{(0, 0, 1) (0, 0, 10), (0, 0, 1) (0, 5, 0), (0, 0, 1) (5, 0, 0), (0, 0, 1) (-5, -5, 0)\}=
		\\
		&= max\{((0)(0) + (0)(0) + (1)(10)), ((0)(0) + (0)(5) + (1)(0)), ((0)(5) + (0)(0) + (1)(0)), ((0)(-5) + (0)(-5) + (1)(0))\}=
		\\
		&=max\{(0+0+10), (0+0+0), (0+0+0), (0+0+0)\}
		=max\{10, 0, 0, 0\} = 10
	\end{align*}
	Pomocou týchto hodnôt teraz vypočítame rozmery našej obálky, keďže vieme, že platí $l_i = MAX_i - MIN_i$
	\begin{align*}
		&l_1^A = MAX_1^A - MIN_1^A = 5 - (-5) = 10
		\\
		&l_2^A = MAX_2^A - MIN_2^A = 5 - (-5) = 10
		\\
		&l_3^A = MAX_3^A - MIN_3^A = 10 - 0 = 10
	\end{align*}
	a podobne o ťažisku obálky vieme, že ho vieme vypočítať ako súčet koeficientami vynásobené vektory ktorými je obal daný čiže: $c=c_1 d_1 + c_2 d_2 + c_3 d_3$. Kde koeficienty sú aritmetické priemery maxima a minima v danom smere: $c_i = \frac{ MAX_i^A + MIN_i^A }{2}$.
	\begin{align*}
		&c_1 = \frac{ MAX_1^A + MIN_1^A }{2} = \frac{ 5 -5 }{2} = \frac{ 0 }{2} = 0
		\\
		&c_2 = \frac{ MAX_2^A + MIN_2^A }{2} = \frac{ 5 -5 }{2} = \frac{ 0 }{2} = 0
		\\
		&c_3 = \frac{ MAX_3^A + MIN_3^A }{2} = \frac{ 10 + 0 }{2} = \frac{ 10 }{2} = 5
		\\
		&c_A = c_1 d_1 + c_2 d_2 + c_3 d_3 = (0)(1, 0, 0) + (0)(0, 1, 0) + (5)(0, 0, 1)  
		= (0, 0, 0) + (0, 0, 0) + (0, 0, 5) = (0, 0, 5) 
	\end{align*}
	
	Teraz tento postup zopakujeme pre objekt B:
	\begin{align*}
		&MIN_1^B = min\{d_1^B B_1, d_1^B B_2, d_1^B B_3, d_1^B B_4\} 
		\\
		&= min\{
		(0.6007367022981083, 0.5080508643267811, 0.6172517588219614) (50, 50, 50), 
		\\&(0.6007367022981083, 0.5080508643267811, 0.6172517588219614) (31, 38, 25), 
		\\&(0.6007367022981083, 0.5080508643267811, 0.6172517588219614) (25, 30, 30), 
		\\&(0.6007367022981083, 0.5080508643267811, 0.6172517588219614) (35, 30, 30)\}=
		\\
		&= min\{
		((0.6007367022981083)(50) + (0.5080508643267811)(50) + (0.6172517588219614)(50)), 
		\\&((0.6007367022981083)(31) + (0.5080508643267811)(38) + (0.6172517588219614)(25)), 
		\\&((0.6007367022981083)(25) + (0.5080508643267811)(30) + (0.6172517588219614)(30)), 
		\\&((0.6007367022981083)(35) + (0.5080508643267811)(30) + (0.6172517588219614)(30))\}=
		\\
		&=min\{
		(30.036835114905415 + 25.40254321633906 + 30.86258794109807), 
		\\&(18.622837771241358 + 19.305932844417683 + 15.431293970549035), 
		\\&(15.018417557452707+ 15.241525929803434 + 18.51755276465884), 
		\\&(21.02578458043379 + 15.241525929803434 + 18.51755276465884)\}=
		\\&=min\{
		86.30196627234254, 
		53.360064586208075, 
		48.77749625191494, 
		54.78486327489607\} = 48.77749625191494
		\\
		&MAX_1^B = max\{d_1^B B_1, d_1^B B_2, d_1^B B_3, d_1^B B_4\} 
		\\
		&= max\{
		(0.6007367022981083, 0.5080508643267811, 0.6172517588219614) (50, 50, 50), 
		\\&(0.6007367022981083, 0.5080508643267811, 0.6172517588219614) (31, 38, 25), 
		\\&(0.6007367022981083, 0.5080508643267811, 0.6172517588219614) (25, 30, 30), 
		\\&(0.6007367022981083, 0.5080508643267811, 0.6172517588219614) (35, 30, 30)\}=
		\\
		&= max\{
		((0.6007367022981083)(50) + (0.5080508643267811)(50) + (0.6172517588219614)(50)), 
		\\&((0.6007367022981083)(31) + (0.5080508643267811)(38) + (0.6172517588219614)(25)), 
		\\&((0.6007367022981083)(25) + (0.5080508643267811)(30) + (0.6172517588219614)(30)), 
		\\&((0.6007367022981083)(35) + (0.5080508643267811)(30) + (0.6172517588219614)(30))\}=
		\\
		&=max\{
		(30.036835114905415 + 25.40254321633906 + 30.86258794109807), 
		\\&(18.622837771241358 + 19.305932844417683 + 15.431293970549035), 
		\\&(15.018417557452707+ 15.241525929803434 + 18.51755276465884), 
		\\&(21.02578458043379 + 15.241525929803434 + 18.51755276465884)\}=
		\\&=max\{
		86.30196627234254, 
		53.360064586208075, 
		48.77749625191494, 
		54.78486327489607\} = 86.30196627234254
		\\
		&MIN_2^B = min\{d_2^B B_1, d_2^B B_2, d_2^B B_3, d_2^B B_4\} 
		\\
		&= min\{
		(0.7991739847383189, -0.40181412653859777, -0.4470641450972119) (50, 50, 50), 
		\\&(0.7991739847383189, -0.40181412653859777, -0.4470641450972119) (31, 38, 25), 
		\\&(0.7991739847383189, -0.40181412653859777, -0.4470641450972119) (25, 30, 30), 
		\\&(0.7991739847383189, -0.40181412653859777, -0.4470641450972119) (35, 30, 30)\}=
		\\
		&= min\{
		((0.7991739847383189)(50) + (-0.40181412653859777)(50) + (-0.4470641450972119)(50)), 
		\\&((0.7991739847383189)(31) + (-0.40181412653859777)(38) + (-0.4470641450972119)(25)), 
		\\&((0.7991739847383189)(25) + (-0.40181412653859777)(30) + (-0.4470641450972119)(30)), 
		\\&((0.7991739847383189)(35) + (-0.40181412653859777)(30) + (-0.4470641450972119)(30))\}=
		\\
		&=min\{
		(39.95869923691595 -20.09070632692989  -22.353207254860592), 
		\\&(24.774393526887888 -15.268936808466716 -11.176603627430296), 
		\\&(19.979349618457974 -12.054423796157932 -13.411924352916357), 
		\\&(27.97108946584116 -12.054423796157932 -13.411924352916357)\}=
		\\&=min\{
		-2.485214344874535, 
		-1.6711469090091242, 
		-5.486998530616315, 
		2.5047413167668715\} = -5.486998530616315
		\\
	\end{align*}
	\begin{align*}
		&MAX_2^B = max\{d_2^B B_1, d_2^B B_2, d_2^B B_3, d_2^B B_4\} 
		\\
		&= max\{
		(0.7991739847383189, -0.40181412653859777, -0.4470641450972119) (50, 50, 50), 
		\\&(0.7991739847383189, -0.40181412653859777, -0.4470641450972119) (31, 38, 25), 
		\\&(0.7991739847383189, -0.40181412653859777, -0.4470641450972119) (25, 30, 30), 
		\\&(0.7991739847383189, -0.40181412653859777, -0.4470641450972119) (35, 30, 30)\}=
		\\
		&= max\{
		((0.7991739847383189)(50) + (-0.40181412653859777)(50) + (-0.4470641450972119)(50)), 
		\\&((0.7991739847383189)(31) + (-0.40181412653859777)(38) + (-0.4470641450972119)(25)), 
		\\&((0.7991739847383189)(25) + (-0.40181412653859777)(30) + (-0.4470641450972119)(30)), 
		\\&((0.7991739847383189)(35) + (-0.40181412653859777)(30) + (-0.4470641450972119)(30))\}=
		\\
		&=max\{
		(39.95869923691595 -20.09070632692989  -22.353207254860592), 
		\\&(24.774393526887888 -15.268936808466716 -11.176603627430296), 
		\\&(19.979349618457974 -12.054423796157932 -13.411924352916357), 
		\\&(27.97108946584116 -12.054423796157932 -13.411924352916357)\}=
		\\&=max\{
		-2.485214344874535, 
		-1.6711469090091242, 
		-5.486998530616315, 
		2.5047413167668715\} = 2.5047413167668715
		\\
		&MIN_3^B = min\{d_3^B B_1, d_3^B B_2, d_3^B B_3, d_3^B B_4\} 
		\\
		&= min\{
		(0.02088915099930236, 0.7618593879259077, -0.6474055270073694) (50, 50, 50), 
		\\&(0.02088915099930236, 0.7618593879259077, -0.6474055270073694) (31, 38, 25), 
		\\&(0.02088915099930236, 0.7618593879259077, -0.6474055270073694) (25, 30, 30), 
		\\&(0.02088915099930236, 0.7618593879259077, -0.6474055270073694) (35, 30, 30)\}=
		\\
		&= min\{
		((0.02088915099930236)(50) + (0.7618593879259077)(50) + (-0.6474055270073694)(50)), 
		\\&((0.02088915099930236)(31) + (0.7618593879259077)(38) + (-0.6474055270073694)(25)), 
		\\&((0.02088915099930236)(25) + (0.7618593879259077)(30) + (-0.6474055270073694)(30)), 
		\\&((0.02088915099930236)(35) + (0.7618593879259077)(30) + (-0.6474055270073694)(30))\}=
		\\
		&=min\{
		(1.0444575499651179+ 38.09296939629539 -32.37027635036847), 
		\\&(0.6475636809783731+ 28.950656741184492 -16.185138175184235), 
		\\&(0.5222287749825589+ 22.85578163777723 -19.42216581022108), 
		\\&(0.7311202849755826+ 22.85578163777723 -19.42216581022108)\}=
		\\&=min\{
		6.7671505958920335, 
		13.413082246978629, 
		3.9558446025387077, 
		4.164736112531731\} = 3.9558446025387077
		\\
		&MAX_3^B = max\{d_3^B B_1, d_3^B B_2, d_3^B B_3, d_3^B B_4\} 
		\\
		&= max\{
		(0.02088915099930236, 0.7618593879259077, -0.6474055270073694) (50, 50, 50), 
		\\&(0.02088915099930236, 0.7618593879259077, -0.6474055270073694) (31, 38, 25), 
		\\&(0.02088915099930236, 0.7618593879259077, -0.6474055270073694) (25, 30, 30), 
		\\&(0.02088915099930236, 0.7618593879259077, -0.6474055270073694) (35, 30, 30)\}=
		\\
		&= max\{
		((0.02088915099930236)(50) + (0.7618593879259077)(50) + (-0.6474055270073694)(50)), 
		\\&((0.02088915099930236)(31) + (0.7618593879259077)(38) + (-0.6474055270073694)(25)), 
		\\&((0.02088915099930236)(25) + (0.7618593879259077)(30) + (-0.6474055270073694)(30)), 
		\\&((0.02088915099930236)(35) + (0.7618593879259077)(30) + (-0.6474055270073694)(30))\}=
		\\
		&=max\{
		(1.0444575499651179+ 38.09296939629539 -32.37027635036847), 
		\\&(0.6475636809783731+ 28.950656741184492 -16.185138175184235), 
		\\&(0.5222287749825589+ 22.85578163777723 -19.42216581022108), 
		\\&(0.7311202849755826+ 22.85578163777723 -19.42216581022108)\}=
		\\&=max\{
		6.7671505958920335, 
		13.413082246978629, 
		3.9558446025387077, 
		4.164736112531731\} = 13.413082246978629
	\end{align*}
	Teraz vypočítame rozmery a ťažisko (podobne ako v prípade objektu A):
	\begin{align*}
		&l_1^B = MAX_1^B - MIN_1^B = 86.30196627234254 - 48.77749625191494 = 37.5244700204276
		\\
		&l_2^B = MAX_2^B - MIN_2^B = 2.5047413167668715 - (-5.486998530616315) = 7.991739847383187
		\\
		&l_3^B = MAX_3^B - MIN_3^B = 13.413082246978629 - 3.9558446025387077 = 9.457237644439921
		\\
		\\
		&c_1 = \frac{ MAX_1^B + MIN_1^B }{2} = \frac{ 86.30196627234254 + 48.77749625191494 }{2} = \frac{ 135.07946252425748 }{2} = 67.53973126212874
		\\
		&c_2 = \frac{ MAX_2^B + MIN_2^B }{2} = \frac{ 2.5047413167668715 -5.486998530616315 }{2} = \frac{ -2.9822572138494436 }{2} = -1.4911286069247218
		\\
		&c_3 = \frac{ MAX_3^B + MIN_3^B }{2} = \frac{ 13.413082246978629 + 3.9558446025387077 }{2} = \frac{ 17.368926849517337 }{2} = 8.684463424758668
		\\
		&c_B = c_1 d_1 + c_2 d_2 + c_3 d_3 = 
		\\
		&=(67.53973126212874)(0.6007367022981083, 0.5080508643267811, 0.6172517588219614) + 
		\\
		&+(-1.4911286069247218)(0.7991739847383189, -0.40181412653859777, -0.4470641450972119) + 
		\\
		&+(8.684463424758668)(0.02088915099930236, 0.7618593879259077, -0.6474055270073694)  
		\\
		&=(40.57359543, 34.31361884, 41.68901791) + 
		\\
		&+(-1.19167119,  0.59915654,  0.66663014) + 
		\\
		&+(0.18141107,  6.61633999, -5.62236962) 
		\\
		&= (39.56333531, 41.52911537, 36.73327843) 
	\end{align*}

	Teraz máme vypočítane už všetky hodnoty a môžeme plnohodnotne definovať obálky našich objektov.
	
	\begin{align*}
		&\text{Objekt A, obálka AABB:}
		\\
		&\text{Smery osí:}
		\\
		&d_1^A = (1, 0, 0) 
		\\
		&d_2^A = (0, 1, 0) 
		\\
		&d_3^A = (0, 0, 1)
		\\
		&\text{Rozmery:}
		\\
		&l_1^A = 10
		\\
		&l_2^A = 10
		\\
		&l_3^A =10
		\\
		&\text{Ťažisko:}
		\\
		&c_A = (0, 0, 5) 
		\\
		\\
		&\text{Objekt B, obálka 0BB:}
		\\
		&\text{Smery osí:}
		\\
		&d_1^B = (0.6007367022981083, 0.5080508643267811, 0.6172517588219614)
		\\
		&d_2^B = (0.7991739847383189, -0.40181412653859777, -0.4470641450972119)
		\\
		&d_3^B = (0.02088915099930236, 0.7618593879259077, -0.6474055270073694) S
		\\
		&\text{Rozmery:}
		\\
		&l_1^B = 37.5244700204276
		\\
		&l_2^B = 7.991739847383187
		\\
		&l_3^B = 9.457237644439921
		\\
		&\text{Ťažisko:}
		\\
		&c_B = (39.56333531, 41.52911537, 36.73327843)
	\end{align*}
	\newpage
	\subsection{}
	Keďže sa nachádzame v 3D priestore a naše objekty sú definované rôznymi osami tak SAT metóda potrebujeme otestovať 15 vektorov (respektíve maximálne 15, ak niektorý dá výsledok, že objekty nie sú v kolízii nemusíme skúšať ďalšie). Tieto vektory sú následovné: (osi objektu A)$d_1^A, d_2^A, d_3^A,$ (osi objektu B)$d_1^B, d_2^B, d_3^B,$ (cross-product všetkých kombinácii os z jedného a druhého objektu)$d_1^A \times d_1^B, d_2^A \times d_1^B, d_3^A \times d_1^B, d_1^A \times d_2^B, d_2^A \times d_2^B, d_3^A \times d_2^B, d_1^A \times d_3^B, d_2^A \times d_3^B, d_3^A \times d_3^B $. (Vzorec pre cross-product ktorý použijeme je: $x \times y = (x_2 y_3 - x_3 y_2, x_3 y_1 - x_1 y_3, x_1 y_2 - x_2 y_1)$)
	\\
	\\
	Jednotlivé vektory:
	\begin{align*}
		&d_1^A = (1, 0, 0) 
		\\
		&d_2^A = (0, 1, 0) 
		\\
		&d_3^A = (0, 0, 1)
		\\
		\\
		&d_1^B = (0.6007367022981083, 0.5080508643267811, 0.6172517588219614)
		\\
		&d_2^B = (0.7991739847383189, -0.40181412653859777, -0.4470641450972119)
		\\
		&d_3^B = (0.02088915099930236, 0.7618593879259077, -0.6474055270073694)
		\\
		\\
		&d_1^A \times d_1^B = (1, 0, 0) \times (0.6007367022981083, 0.5080508643267811, 0.6172517588219614) =
		\\
		&= ((0) (0.6172517588219614) - (0) (0.5080508643267811), 
		\\
		&(0) (0.6007367022981083) - (1) (0.6172517588219614), 
		\\
		&(1) (0.5080508643267811) - (0) (0.6007367022981083)) =
		\\
		&=(0-0, 0-0.6172517588219614, 0.5080508643267811-0)=
		\\
		&=(0, -0.6172517588219614, 0.5080508643267811)
		\\
		\\
		&d_2^A \times d_1^B = (0, 1, 0) \times (0.6007367022981083, 0.5080508643267811, 0.6172517588219614)
		\\
		&= ((1) (0.6172517588219614) - (0) (0.5080508643267811), 
		\\
		&(0) (0.6007367022981083) - (0) (0.6172517588219614), 
		\\
		&(0) (0.5080508643267811) - (1) (0.6007367022981083)) =
		\\
		&=(0.6172517588219614-0, 0-0, 0-0.6007367022981083)=
		\\
		&=(0.6172517588219614, 0, -0.6007367022981083)
		\\
		\\
		&d_3^A \times d_1^B = (0, 0, 1) \times (0.6007367022981083, 0.5080508643267811, 0.6172517588219614)
		\\
		&= ((0) (0.6172517588219614) - (1) (0.5080508643267811), 
		\\
		&(1) (0.6007367022981083) - (0) (0.6172517588219614), 
		\\
		&(0) (0.5080508643267811) - (0) (0.6007367022981083)) =
		\\
		&=(0-0.5080508643267811, 0.6007367022981083-0, 0-0)=
		\\
		&=(-0.5080508643267811, 0.6007367022981083, 0)
		\\
		\\
		&d_1^A \times d_2^B = (1, 0, 0) \times (0.7991739847383189, -0.40181412653859777, -0.4470641450972119)
		\\
		&= ((0) (-0.4470641450972119) - (0) (-0.40181412653859777), 
		\\
		&(0) (0.7991739847383189) - (1) (-0.4470641450972119), 
		\\
		&(1) (-0.40181412653859777) - (0) (0.7991739847383189)) =
		\\
		&=(0-0, 0-(-0.4470641450972119), (-0.40181412653859777)-0)=
		\\
		&=(0, 0.4470641450972119, -0.40181412653859777)
	\end{align*}
	\begin{align*}
		&d_2^A \times d_2^B = (0, 1, 0) \times (0.7991739847383189, -0.40181412653859777, -0.4470641450972119)
		\\
		&= ((1) (-0.4470641450972119) - (0) (-0.40181412653859777), 
		\\
		&(0) (0.7991739847383189) - (0) (-0.4470641450972119), 
		\\
		&(0) (-0.40181412653859777) - (1) (0.7991739847383189)) =
		\\
		&=((-0.4470641450972119)-0, 0-0, 0-0.7991739847383189)=
		\\
		&=(-0.4470641450972119, 0, -0.7991739847383189)
		\\
		\\
		&d_3^A \times d_2^B = (0, 0, 1) \times (0.7991739847383189, -0.40181412653859777, -0.4470641450972119)
		\\
		&= ((0) (-0.4470641450972119) - (1) (-0.40181412653859777), 
		\\
		&(1) (0.7991739847383189) - (0) (-0.4470641450972119), 
		\\
		&(0) (-0.40181412653859777) - (0) (0.7991739847383189)) =
		\\
		&=(0-(-0.40181412653859777), 0.7991739847383189-0, 0-0)=
		\\
		&=(0.40181412653859777, 0.7991739847383189, 0)
		\\
		\\
		&d_1^A \times d_3^B = (1, 0, 0) \times (0.02088915099930236, 0.7618593879259077, -0.6474055270073694)
		\\
		&= ((0) (-0.6474055270073694) - (0) (0.7618593879259077), 
		\\
		&(0) (0.02088915099930236) - (1) (-0.6474055270073694), 
		\\
		&(1) (0.7618593879259077) - (0) (0.02088915099930236)) =
		\\
		&=(0-0, 0-(-0.6474055270073694), (0.7618593879259077)-0)=
		\\
		&=(0, 0.6474055270073694, 0.7618593879259077)
		\\
		\\
		&d_2^A \times d_3^B = (0, 1, 0) \times (0.02088915099930236, 0.7618593879259077, -0.6474055270073694)
		\\
		&= ((1) (-0.6474055270073694) - (0) (0.7618593879259077), 
		\\
		&(0) (0.02088915099930236) - (0) (-0.6474055270073694), 
		\\
		&(0) (0.7618593879259077) - (1) (0.02088915099930236)) =
		\\
		&=((-0.6474055270073694)-0, 0-0, 0-0.02088915099930236)=
		\\
		&=(-0.6474055270073694, 0, -0.02088915099930236)
		\\
		\\
		&d_3^A \times d_3^B = (0, 0, 1) \times (0.02088915099930236, 0.7618593879259077, -0.6474055270073694)
		\\
		&= ((0) (-0.6474055270073694) - (1) (0.7618593879259077), 
		\\
		&(1) (0.02088915099930236) - (0) (-0.6474055270073694), 
		\\
		&(0) (0.7618593879259077) - (0) (0.02088915099930236)) =
		\\
		&=(0-0.7618593879259077, 0.02088915099930236-0, 0-0)=
		\\
		&=(-0.7618593879259077, 0.02088915099930236, 0)
	\end{align*}
	\newpage
	\subsection{}
	Keďže ideme počítať priemet spojnice ťažísk potrebujeme si najprv vypočítať vektor spájajúci ťažiská (jeho veľkosť bude vzdialenosť medzi nimi). Tento vektor označíme $r_{AB}$ a vypočíta sa takto:
	\\
	\begin{align*}
		&r_{AB} = c_B - c_A = (39.56333531, 41.52911537, 36.73327843) - (0, 0, 5) =
		\\
		&= (39.56333531-0, 41.52911537-0, 36.73327843-5) =
		\\
		&= (39.56333531, 41.52911537, 31.73327843)
	\end{align*}
	Teraz môžeme vypočítať projekcie spojnice na vektory $d_1^B$, $d_1^A \times d_2^B$ a $d_2^A \times d_3^B$, konkrétne to čo nás zaujíma je veľkosť projekcie spojnice preto to čo budeme počítať je $s = |v r|$:
	\begin{align*}
		&s_1^{AB} = |d_1^B r_{AB} | = 
		\\
		&= | (0.6007367022981083, 0.5080508643267811, 0.6172517588219614) (39.56333531, 41.52911537, 31.73327843) | =
		\\
		&=| (0.6007367022981083)(39.56333531) + (0.5080508643267811)(41.52911537) + (0.6172517588219614)(31.73327843)  | =
		\\
		&= | 23.767147586043706+ 21.09890295845511+ 19.587421924104508 | =
		\\
		&= 64.45347246860332
		\\
		\\
		&s_2^{AB} = |(d_1^A \times d_2^B) r_{AB} | = 
		\\
		&= | (0, 0.4470641450972119, -0.40181412653859777) (39.56333531, 41.52911537, 31.73327843) | =
		\\
		&=| (0)(39.56333531) + (0.4470641450972119)(41.52911537) + (-0.40181412653859777)(31.73327843)  | =
		\\
		&= | 0+18.56617845953253 -12.750879554556574| =
		\\
		&= 5.815298904975956
		\\
		\\
		&s_3^{AB} = |(d_2^A \times d_3^B) r_{AB} | = 
		\\
		&= | (-0.6474055270073694, 0, -0.02088915099930236) (39.56333531, 41.52911537, 31.73327843) | =
		\\
		&=| (-0.6474055270073694)(39.56333531) + (0)(41.52911537) + (-0.02088915099930236)(31.73327843)  | =
		\\
		&= | -25.613521946539816+ 0-0.6628812448271745 | =
		\\
		&= 26.27640319136699
		\\
	\end{align*} 
	\newpage
	\subsection{}
	Podobne ako sme počítali projekciu spojnice ťažísk tak nás zaujíma aj projekcia samotných objektov. Znovu nás zaujíma iba veľkosť tohto priemetu, preto budeme tento priemet počítať ako súčet priemetov jednotlivých vektor ktorými je objekt daný (vlastných vektorov) pričom každý priemet vynásobíme veľkosťou rozmeru objektu v danom smere (viem, že by sme mali asi násobiť polovicou rozmeru ale osobne sa mi zdá rozumnejšie vypočítať priemet celého objektu a následne pri testovaní kolízie na základe priemetov počítať iba s polovicami objektov medzi ťažiskami): $h=l_1|vd_1| +l_2|vd_2|+l_3|vd_3|$. Pričom znovu budeme premietať objekty na $d_1^B$, $d_1^A \times d_2^B$ a $d_2^A \times d_3^B$.
	
	Začnime najskôr s projekciami objektu A:
	\begin{align*}
		&h_1^A = L_1^A|d_1^B d_1^A | + L_2^A|d_1^B d_2^A | + L_3^A|d_1^B d_3^A | = 
		\\
		&= 10| (0.6007367022981083, 0.5080508643267811, 0.6172517588219614) (1,0,0) |+
		\\
		&+ 10| (0.6007367022981083, 0.5080508643267811, 0.6172517588219614) (0,1,0) |+
		\\
		&+ 10| (0.6007367022981083, 0.5080508643267811, 0.6172517588219614) (0,0,1) |=
		\\
		&=10 | (0.6007367022981083)(1) + (0.5080508643267811)(0) + (0.6172517588219614)(0)  |+
		\\
		& +10 | (0.6007367022981083)(0) + (0.5080508643267811)(1) + (0.6172517588219614)(0)  | +
		\\
		& +10 | (0.6007367022981083)(0) + (0.5080508643267811)(0) + (0.6172517588219614)(1)  | 
		\\
		&= 10| 0.6007367022981083 + 0 + 0 | +
		10| 0 + 0.5080508643267811 + 0 |+
		10| 0 + 0 + 0.6172517588219614 |=
		\\
		&= 10| 0.6007367022981083 | +
		10| 0.5080508643267811 |+
		10| 0.6172517588219614 |=
		\\
		&=6.007367022981083 +
		5.080508643267811+
		6.172517588219614 =
		\\
		&= 17.26039325446851
		\\
		\\
		&h_2^A = L_1^A|(d_1^A \times d_2^B) d_1^A | + L_2^A|(d_1^A \times d_2^B) d_2^A | + L_3^A|(d_1^A \times d_2^B) d_3^A | = 
		\\
		&= 10| (0, 0.4470641450972119, -0.40181412653859777) (1,0,0) |+
		\\
		&+ 10| (0, 0.4470641450972119, -0.40181412653859777) (0,1,0) |+
		\\
		&+ 10| (0, 0.4470641450972119, -0.40181412653859777) (0,0,1) |=
		\\
		&=10 | (0)(1) + (0.4470641450972119)(0) + (-0.40181412653859777)(0)  |+
		\\
		& +10 | (0)(0) + (0.4470641450972119)(1) + (-0.40181412653859777)(0)  | +
		\\
		& +10 | (0)(0) + (0.4470641450972119)(0) + (-0.40181412653859777)(1)  | 
		\\
		&= 10| 0 + 0 + 0 | +
		10| 0 + 0.4470641450972119 + 0 |+
		10| 0 + 0 -0.40181412653859777 |=
		\\
		&= 10| 0 | +
		10| 0.4470641450972119 |+
		10| -0.40181412653859777 |=
		\\
		&=0 +
		4.470641450972119+
		4.0181412653859777 =
		\\
		&= 8.488782716358097
		\\
		\\
		&h_3^A = L_1^A|(d_2^A \times d_3^B) d_1^A | + L_2^A|(d_2^A \times d_3^B) d_2^A | + L_3^A|(d_2^A \times d_3^B) d_3^A | = 
		\\
		&= 10| (-0.6474055270073694, 0, -0.02088915099930236) (1,0,0) |+
		\\
		&+ 10| (-0.6474055270073694, 0, -0.02088915099930236) (0,1,0) |+
		\\
		&+ 10| (-0.6474055270073694, 0, -0.02088915099930236) (0,0,1) |=
		\\
		&=10 | (-0.6474055270073694)(1) + (0)(0) + (-0.02088915099930236)(0)  |+
		\\
		& +10 | (-0.6474055270073694)(0) + (0)(1) + (-0.02088915099930236)(0)  | +
		\\
		& +10 | (-0.6474055270073694)(0) + (0)(0) + (-0.02088915099930236)(1)  | 
		\\
		&= 10| -0.6474055270073694 + 0 + 0 | +
		10| 0 + 0 + 0 |+
		10| 0 + 0 -0.02088915099930236 |=
		\\
		&= 10| -0.6474055270073694 | +
		10| 0 |+
		10| -0.02088915099930236 |=
		\\
		&=6.474055270073694 +
		0+
		0.2088915099930236 =
		\\
		&= 6.682946780066717
	\end{align*}
	\newpage
	Teraz vypočítajme projekcie objektu B (niektoré jednoduché kroky budem preskakovať keďže už som na 19, strane a som unavený z písanie každého jedného kroku výpočtu).
	\begin{align*}
		&h_1^B = L_1^B|d_1^B d_1^B | + L_2^B|d_1^B d_2^B | + L_3^B|d_1^B d_3^B | = 
		\\
		&= 37.5244700204276| (0.6007367022981083, 0.5080508643267811, 0.6172517588219614)\\& (0.6007367022981083, 0.5080508643267811, 0.6172517588219614) |+
		\\
		&+ 7.991739847383187| (0.6007367022981083, 0.5080508643267811, 0.6172517588219614)\\& (0.7991739847383189, -0.40181412653859777, -0.4470641450972119) |+
		\\
		&+ 9.457237644439921| (0.6007367022981083, 0.5080508643267811, 0.6172517588219614)\\& (0.02088915099930236, 0.7618593879259077, -0.6474055270073694) |=
		\\
		&=37.5244700204276 | (0.6007367022981083)(0.6007367022981083) +\\&+ (0.5080508643267811)(0.5080508643267811) +\\&+ (0.6172517588219614)(0.6172517588219614)  |+
		\\
		& +7.991739847383187 | (0.6007367022981083)(0.7991739847383189) +\\&+ (0.5080508643267811)(-0.40181412653859777) +\\&+ (0.6172517588219614)(-0.4470641450972119)  | +
		\\
		& +9.457237644439921 | (0.6007367022981083)(0.02088915099930236) +\\&+ (0.5080508643267811)(0.7618593879259077) +\\&+ (0.6172517588219614)(-0.6474055270073694)  | 
		\\
		&= 37.5244700204276| 1 | +
		7.991739847383187| 0 |+
		9.457237644439921| 0 |=
		\\
		&=37.5244700204276 +
		0 +
		0 =
		\\
		&= 37.5244700204276
		\\
		&h_2^B = L_1^B|(d_1^A \times d_2^B) d_1^B | + L_2^B|(d_1^A \times d_2^B) d_2^B | + L_3^B|(d_1^A \times d_2^B) d_3^B | = 
		\\
		&= 37.5244700204276| (0, 0.4470641450972119, -0.40181412653859777)\\& (0.6007367022981083, 0.5080508643267811, 0.6172517588219614) |+
		\\
		&+ 7.991739847383187| (0, 0.4470641450972119, -0.40181412653859777) \\&(0.7991739847383189, -0.40181412653859777, -0.4470641450972119) |+
		\\
		&+ 9.457237644439921| (0, 0.4470641450972119, -0.40181412653859777)\\& (0.02088915099930236, 0.7618593879259077, -0.6474055270073694) |=
		\\
		&=37.5244700204276 | (0)(0.6007367022981083) +\\&+ (0.4470641450972119)(0.5080508643267811) +\\&+ (-0.40181412653859777)(0.6172517588219614)  |+
		\\
		& +7.991739847383187 | (0)(0.7991739847383189) +\\&+ (0.4470641450972119)(-0.40181412653859777) +\\&+ (-0.40181412653859777)(-0.4470641450972119)  | +
		\\
		& +9.457237644439921 | (0)(0.02088915099930236) +\\&+ (0.4470641450972119)(0.7618593879259077) +\\&+ (-0.40181412653859777)(-0.6474055270073694)  | 
		\\
		&= 37.5244700204276| -0.020889150999307632 | +
		7.991739847383187| 0 |+
		9.457237644439921| 0.6007367022981078 |=
		\\
		&=0.7838543204257045 +
		0 +
		5.681309755370363 =
		\\
		&=6.465164075796068 
		\\
	\end{align*}
	\begin{align*}
		&h_3^B = L_1^B|(d_2^A \times d_3^B) d_1^B | + L_2^B|(d_2^A \times d_3^B) d_2^B | + L_3^B|(d_2^A \times d_3^B) d_3^B | = 
		\\
		&= 37.5244700204276| (-0.6474055270073694, 0, -0.02088915099930236) \\&(0.6007367022981083, 0.5080508643267811, 0.6172517588219614) |+
		\\
		&+ 7.991739847383187| (-0.6474055270073694, 0, -0.02088915099930236)\\&(0.7991739847383189, -0.40181412653859777, -0.4470641450972119) |+
		\\
		&+ 9.457237644439921| (-0.6474055270073694, 0, -0.02088915099930236)\\& (0.02088915099930236, 0.7618593879259077, -0.6474055270073694) |=
		\\
		&=37.5244700204276 | (-0.6474055270073694)(0.6007367022981083) +\\&+ (0)(0.5080508643267811) +\\&+ (-0.02088915099930236)(0.6172517588219614)  |+
		\\
		& +7.991739847383187 | (-0.6474055270073694)(0.7991739847383189) +\\&+ (0)(-0.40181412653859777) +\\&+ (-0.02088915099930236)(-0.4470641450972119)  | +
		\\
		& +9.457237644439921 | (-0.6474055270073694)(0.02088915099930236) +\\&+ (0)(0.7618593879259077) +\\&+ (-0.02088915099930236)(-0.6474055270073694)  | 
		\\
		&= 37.5244700204276| -0.40181412653859294 | +
		7.991739847383187| -0.5080508643267811 |+
		9.457237644439921| 0 |=
		\\
		&=15.077862145081733 +
		4.060210336937806 +
		0 =
		\\
		&= 19.13807248201954
	\end{align*}

	\newpage
	\subsection{}
	Teraz keď máme vypočítane projekcie spojnice ťažísk aj objektov môžeme testovať či niektorá z testovaných osí je deliacou. Na to aby aby mohla byť os deliacou musí platiť, že veľkosť priemetu spojnice ťažníc je väčšia ako súčet veľkostí polovíc priemetov objektov (polovíc preto lebo iba polovica objektu sa nachádza od ťažnice v smere spojnice ťažníc) takže musí platiť: $s_{AB}>\frac{h_A}{2} + \frac{h_B}{2} $. 
	
	Testovanie deliacich osí:
	\begin{align*}
		&\text{Os $d_1^B$:}
		\\
		&s_1^{AB} = 64.45347246860332
		\\
		&\frac{h_1^A}{2} + \frac{h_1^B}{2} = \frac{17.26039325446851}{2} + \frac{37.5244700204276}{2}=
		8.630196627234255 + 18.7622350102138 =
		27.392431637448055
		\\
		&s_1^{AB} > \frac{h_1^A}{2} + \frac{h_1^B}{2}
		\\
		&\text{Os $d_1^B$ je deliacou}
		\\
		\\
		&\text{Os $d_1^A \times d_2^B$:}
		\\
		&s_2^{AB} = 5.815298904975956
		\\
		&\frac{h_2^A}{2} + \frac{h_2^B}{2} = \frac{8.488782716358097}{2} + \frac{6.465164075796068 }{2}=
		4.2443913581790484 + 3.232582037898034 =
		7.476973396077082
		\\
		&s_1^{AB} < \frac{h_1^A}{2} + \frac{h_1^B}{2}
		\\
		&\text{Os $d_1^A \times d_2^B$ nie je deliacou}
		\\
		\\
		&\text{Os $d_2^A \times d_3^B$:}
		\\
		&s_2^{AB} = 26.27640319136699
		\\
		&\frac{h_2^A}{2} + \frac{h_2^B}{2} = \frac{6.682946780066717}{2} + \frac{19.13807248201954 }{2}=
		3.3414733900333586 + 9.56903624100977 =
		12.910509631043128
		\\
		&s_1^{AB} > \frac{h_1^A}{2} + \frac{h_1^B}{2}
		\\
		&\text{Os $d_2^A \times d_3^B$ je deliacou}
		\\
	\end{align*}
	
	Vidíme, že hneď pre dve z nami testovaných osí sme zistili, že ide deliacu os vďaka tomu môžeme tvrdiť, že objekty A a B nie sú v kolízii. 
\end{document}